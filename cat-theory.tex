\section{Category Theory}

\begin{definition} 
Let $\Cat{A}$ be an abelian category. A subcategory
$\Cat{C}$ is a \emph{Serre subcategory of $\Cat{A}$}
if
\begin{enumerate}
\item $\Cat{C}$ is a full subcategory of $\Cat{A}$

\item If $0 \to A' \to A \to A''\to 0$ is exact in $\Cat{A}$
then $A \in \Cat{C}$ if and only if $A'$ and $A''$ are in
$\Cat{C}$

\item $\Cat{C}$ is nonempty.
\end{enumerate}
\end{definition}

Example 3 is of great interest to us: Let $T$ be an exact
functor from $\Cat{A}$ to $\Cat{B}$, both abelian categories.
Let $\Cat{C}$ be the full subcategory of $\Cat{A}$ with objects
equal to $\{A | T(A) = 0\}$. Then $\Cat{C}$ is a Serre 
subcategory.

The proof here is obvious: $0 \in \Cat{C}$, which is by 
definition full. Given
\[
0 \to A' \to A \to A'' \to 0
\]
of objects in $\Cat{A}$. Then, the following is exact:
\[
0 \to T(A') \to T(A) \to T(A'') \to 0.
\]
It is clear that $A \in \Cat{C}$ if and only if $A'$ and
$A''$ are members of $\Cat{C}$.

\begin{definition}
A category $\Cat{A}$ is \emph{well-powered} if the poset
of ``subobjects'' of any object $A \in \Cat{A}$ is small.
Here, ``subobject'' refers to an equivalence class of
monomorphisms.
\end{definition}

The comment ``Serre has given...'' refers to Theorem 2.1 on
page 44, where the category $\Cat{A}/\Cat{C}$ is constructed
for every abelian category $\Cat{A}$ and Serre subcategory
$\Cat{C}$.

\begin{prop}
A nonempty full subcategory $\Cat{C}$ of an abelian category
$\Cat{A}$ is a Serre subcategory if and only if for every
exact sequence
\[
A' \stackrel{f}{\to} A \stackrel{g}{\to} A''
\]
in $\Cat{A}$ with $A'$ and $A''$ in $\Cat{C}$ implies
$A \in \Cat{C}$.
\end{prop}
\begin{proof}
\noindent$\Leftarrow$: Show that $0 \in \Cat{C}$, and then show
use $0 \to A' \to A$ and $A \to A'' \to 0$ to show that $A'$
and $A''$ are in $\Cat{C}$

\noindent$\Rightarrow$: For
\[
A' \stackrel{f}{\to} A \stackrel{g}{\to} A''
\]
show that $\im f = \ker g$ and $\im g$ are all
objects of $\Cat{C}$ and use
\[
0 \to \ker g \to A \to im g \to 0.
\]
\end{proof}

\begin{definition}
Let $\Cat{C}$ be a Serre subcategory of $\Cat{A}$ and let
$f : A \to B$ in $\Cat{A}$. Then
\begin{itemize}
\item $f$ is a \emph{$\Cat{C}$-mono} if $\ker f \in \Cat{C}$
\item $f$ is a \emph{$\Cat{C}$-epi} if $\cok f \in \Cat{C}$
\item $f$ is a \emph{$\Cat{C}$-iso} if $f$ is both a mono and
      an epi
\end{itemize}
\end{definition}

\begin{lem}
Let $A \stackrel{f}{\to} B \stackrel{g}{\to} C$ be an exact
sequence in $\Cat{A}$. 
\begin{enumerate}
\item If $f$ and $g$ are $\Cat{C}$-*, then so is $gf$.

\item If $gf$ is $\Cat{C}$-mono, then $f$ is $\Cat{C}$-mono.

\item If $f$ is $\Cat{C}$-iso, then $g$ is a $\Cat{C}$-* if
      and only if $gf$ is.

\item If $g$ is $\Cat{C}$-iso, then $f$ is a $\Cat{C}$-* if
      and only if $gf$ is.
\end{enumerate}
\end{lem}
\begin{proof}
The proof uses the 3-Lemma; stated simply: if 
$A \stackrel{f}{\to} B \stackrel{g}{\to} C$
is exact, then so is
\[
0 \to \ker f \to \ker gf \to \ker g \to \cok f \to \cok gf \to
\cok g \to 0,
\]
from which the lemma follows from Prop 1.1.

To see this, apply the Snake Lemma to
\[
\begin{diagram}
0 &\rTo &A &\rTo{\id \oplus f} &A \oplus B &\rTo{f - \id} &B &\rTo &0 \\
  &   &\dTo{f} &   &\dTo{gf \oplus id} &          &\dTo{g}  &   & \\
0 &\rTo &B &\rTo{g\oplus \id} &C \oplus B &\rTo{\id - g} &C &\rTo &0 
\end{diagram}
\]
\end{proof}

\begin{definition}
Let $\Cat{A}$ be an abelian category, $\Cat{C}$ a Serre subcategory,
and $L \in \Cat{A}$ is \emph{$\Cat{C}$-closed} if for $\Cat{C}$-iso 
$u: A \to B$,
\[
\hom(B, L) \to \hom(A, L) 
\]
is an isomorphism.
\end{definition}

\begin{lem}
$L$ is $\Cat{C}$-closed if and only if
\begin{enumerate}
\item $L$ has no $\Cat{C}$ subobject. That is, $C \in \Cat{C}$ 
      implies $\hom(C, L) = 0$, and

\item If $0 \to L \to X \to C \to 0$ is exact, and $C \in \Cat{C}$
      then the sequence splits.
\end{enumerate}
\end{lem}
\begin{proof}
\noindent $\Rightarrow$: Use the $\Cat{C}$-iso $C \to 0$, and
note that the map $L \to X$ in (2) is a $\Cat{C}$-iso.

\noindent $\Leftarrow$: Consider $f: A \to B$, and reduce to the
case where $f$ is either an mono or an epi by considering $\ker f$, 
$\im f$ and $\cok f$. 

The case where $f$ is epi, complete to a s.e.s, and note that the
kernel is in $\Cat{C}$; use (1).

The case where $f$ is mono, conclude $f$ induces injection on homs 
via same reasoning as epi; consider push-out $P$ of $g: A \to L$ and
$f: A \to B$, and complete the square into map between s.e.s's:
\[
\begin{diagram}
0 & \rTo &   A     &\rTo{f}  & B                &  \rTo     & C       &\rTo & 0 \\
  &      &\dTo{g}  &        &\dTo{g'}          &          &\dTo{g}  &    & \\
0 & \rTo &   L     &\rTo{f'} & P                &  \rTo     & C'      &\rTo & 0.
\end{diagram}
\]
Using the universal property of $P$, via $B \to C$ and $L 
\stackrel{0}{\to} C$, we get $P \to C$; $C' \to C$ comes
from the universal property of cokernels. Show that $C \simeq 
C'$, so $C' \in \Cat{C}$. Finish by splitting the bottom s.e.s. 
using (2).
\end{proof}

\begin{definition}
$u : A \to L$ is called a \emph{$\Cat{C}$-envelop of $A$} if $u$ is a
$\Cat{C}$-iso and $L$ is $\Cat{C}$-closed.
\end{definition}

\begin{definition}
A Serre subcategory $\Cat{C}$ of $\Cat{A}$ is a \emph{localizing 
subcategory} if every object in $\Cat{A}$ has a $\Cat{C}$-envelop.
\end{definition}

\begin{rmk}
It is worth mentioning that Gabriel defined localizing by 
considering $T: \Cat{A} \to \Cat{A}/\Cat{C}$: $\Cat{C}$ is 
localizing if and only if $T$ admits a left adjoint.
\end{rmk}

\begin{thm}
Let $\Cat{C}$ be a localizing subcategory of $\Cat{A}$. Let
$\Cat{L}$ be the full subcategory of all $\Cat{C}$-closed objects.
Then
\begin{enumerate}
\item $\Cat{L}$ is reflexive. That is, the inclusion $i: \Cat{L}
\to \Cat{A}$ has a left adjoint $R: \Cat{A} \to \Cat{L}$.

\item $\Cat{L}$ is abelian (but can fail to be an exact subcateory
of $\Cat{A}$).

\item $R$ is an exact functor from $\Cat{A}$ to $\Cat{A}$.

\item If $\Cat{A}$ is left (resp. right) complete, then so is
$\Cat{L}$.

\end{enumerate}
\end{thm}
