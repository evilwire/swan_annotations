\section{Localization}

\noindent For the following, let $\Cat{A}$ be an abelian 
category, and let $\Cat{C}$ be a Serre subcategory. The following 
sections will deal with the following two questions:

\noindent {\bf Problem 1.} Find an exact functor $T$ and an 
abelian category $\Cat{A}/\Cat{C}$ with $T: \Cat{A} \to 
\Cat{A}/\Cat{C}$ such that if $F: \Cat{A} \to \Cat{B}$ is an 
exact functor which annihilates $\Cat{C}$ then there exists a 
unique functor $G$ with $F = GT$; $G$ will be exact. This is 
possible if $\Cat{A}$ is well-powered (proven in the next 
section) or if $\Cat{C}$ is localizing.

\vskip 10pt

\noindent {\bf Problem 2.} Find an exact functor $T$ from 
$\Cat{A}$ to some category $\Cat{D}$ which is universal for exact 
functors which annihilate $\Cat{C}$ up to natural isomophism. 
That is, given the diagram
\[
\begin{diagram}
   &          & D           \\
   & \ruTo{T} & \dDashto{G} \\
 A \\
   & \rdTo{F} \\
   &          & B
\end{diagram}
\]
where $F$ is exact and annihilates $C$. Then $G: \Cat{D} \to
\Cat{B}$ is exact and $GT$ is naturally isomorphic to $F$
and any other $G'$ is naturally isomorphic to $G$.

We solve the problems first for $\Cat{C}$ localizing.

\vskip 10pt
\begin{prop} \label{prop_2_0}
If $\Cat{C}$ is localizing, then $\Cat{L}$ and the reflection
$R: \Cat{A} \to \Cat{L}$ solves Problem 2.
\end{prop}
\begin{proof}
Let $F: \Cat{A} \to \Cat{B}$ be exact and annihilate $\Cat{C}$.
Set $G = F|_{\Cat{L}}$. Certainly the following commutes:
\[
\begin{diagram}
   &          & \Cat{L}     \\
   & \ruTo{T} & \dDashto{G} \\
 A \\
   & \rdTo{F} \\
   &          & B
\end{diagram}
\]
We must show that $G$ is exact. For s.e.s. in $\Cat{L}$,
\[
0 \to L' \to L \to L'' \to 0
\]
we have
\[
0 \to L' \to L \to L'' \to C \to 0
\]
in $\Cat{A}$, and applying $R$, note that $C \in \Cat{C}$;
but $\Cat{C}$ is annihilated by $F$. Therefore, $FC = 0$,
and $G = F|_{\Cat{L}}$ is exact.

Next, we show $GR$ is naturally iso to $F$. For each $A \in
\Cat{A}$, we have a $\Cat{C}$-iso $u_A : A \to RA$, which
induces an iso $FA \to FRA$ which is natural in $A$. As
$FRA = GRA$, $FA \to GRA$ is a natural iso.

Finally, we show that $F|_{\Cat{C}}$ is universal. Fix a
natural iso $\nu : GR \to F$. Then $\nu|_{\Cat{L}}$ is a
natural isomorphism from $GR|_{\Cat{L}} \to F|_{\Cat{L}}$.
But $GR|_{\Cat{L}} = G$.
\end{proof}

\begin{rmk}
I have yet to give a satisfactory explanation to the following 
passage, taken directly from Swan's notes:

\begin{quotation}
It could happen that $\Cat{A}$ was very big and $\Cat{L}$ small 
(by creating a new equivalent category with lots of objects in 
each is isomorphism type of $\Cat{A}$ not in $\Cat{L}$). Then the 
above construction won't work for problem 1 because $R$ will 
identify too many objects to be able to factor $F$ as $GR$. This
is the only difficulty. We remove it by the following construction.
\end{quotation}

My thoughts: Consider $A$ and its $\Cat{C}$-envelope $RA$; by 
definition, $RRA = RA$, and therefore $A$ necessarily maps to the 
same object as $RA$ via $GR$. Conversely, $FA$ may not be 
identified as $FRA$. Thus, necessarily, $GR \neq F$.

To make the above discussion concrete, we produce the following
$F$ as an example of the above, and also a solution to Problem 1.
\end{rmk}

Let $\AC$ be the category whose objects are objects of $\Cat{A}$
and 
\[
\hom_{\AC}(A, B) = \hom_{\Cat{A}}(RA, RB).
\]
There is a functor $T: \Cat{A} \to \AC$ defined by $TA = A$ and 
$Tf = Rf$. Let $S : \AC \to \Cat{L}$ be the functor given by $A 
\mapsto RA$ and $Sf = f$. It is clear that $R$ is an equivalence 
with inverse $S$.

It follows that $\AC$ is an abelian category that annihilates
$\Cat{C}$. Indeed, for $A \in \Cat{A}$, $RA = 0$ if and only if
$A \in \Cat{C}$. We also have the following:

\begin{thm}
The category $\AC$ defined as above solves problem 1 for 
$\Cat{A}$ with a localizing Serre subcategory $\Cat{C}$.
\end{thm}

\begin{proof}
$T$ is exact since $R$ is exact.

To show that $T$ is universal, consider $F : \Cat{A} \to \Cat{B}$
an exact functor that annihilates $\Cat{C}$, we want $G$ such that
$GT = F$. If such a $G$ exists, then it will be exact since $GSR$
is naturally iso to $F$, and $GS$ is exact by Proposition 
\ref{prop_2_0}; but $S$ is an equivalence.

Define $G = F$ on objects.

Next, consider $f \in \hom_{\AC}(A, B)$. This corresponds to a 
diagram in $\Cat{A}$:
\begin{equation}\label{2_star}
\begin{diagram}
A         &         & B         \\
\dTo{u_A} &         & \dTo{u_B} \\
RA        & \rTo{f} & RB
\end{diagram}
\end{equation}
Applying $F$ to the above, we have
\[
\begin{diagram}
FA         &          & FB         \\
\dTo{Fu_A} &          & \dTo{Fu_B} \\
FRA        & \rTo{Ff} & FRB
\end{diagram}
\]
Note that $Fu_A, Fu_B$ are isos, since $u_A, u_B$ are 
$\Cat{C}$-isos. Set 
\[
G(f) = (Tu_B)^{-1}F(f)Tu_A.
\]
First, consider
$u_A : A \to RA$. We want to show that 
\[
GTu_A = Fu_A.
\]
To see this, notice that $RA = RRA$ so $Ru_A = \id_{RA} = 
R\id_{RA}$. We have the following diagram associated to
$Tu_A$:
\[
\begin{diagram}
A         &            & RA             \\
\dTo{u_A} &            & \dTo{\id_{RA}} \\
RA        & \rTo{Tu_A} & RRA
\end{diagram}
\]
From which we derive
\[
\begin{diagram}
FA         &           & FRA       \\
\dTo{Fu_A} &           & \dTo{\id} \\
FRA        & \rTo{\id} & FRA
\end{diagram}
\]
And it is clear that $GTu_A = Fu_A$.

And similarly, applying $T$ to (\ref{2_star}), we have
\[
\begin{diagram}
TA         & \rTo{f}  & TB         \\
\dTo{Tu_A} &          & \dTo{Tu_B} \\
TRA        & \rTo{Tf} & TRB
\end{diagram}
\]

Applying $G$ to the above, we have
\[
\begin{diagram}
FA         & \rTo{Gf}  & FB         \\
\dTo{Fu_A} &           & \dTo{Fu_B} \\
FRA        & \rTo{GTf} & FRB
\end{diagram}
\]
and since $Fu_A$ and $Fu_B$ are isomorphisms, we have 
\[
GTf = Fu_BGf(Fu_A)^{-1} = Ff.
\]
Similar argumens show that $GTf = Ff$
for any $f \in \Cat{A}$.
\end{proof}

\begin{definition}
Fix $\Cat{A}$ an abelian category, and let $\Cat{C}$ be a Serre
subcategory. Suppose $\AC$ is any category with an exact functor 
$T: \Cat{A} \to \AC$. 

We say that $\AC$ is a \emph{weak localization of $\Cat{A}$ by 
$\Cat{C}$} exact functors annihilating $\Cat{C}$ (i.e. $T$ solves 
Problem 2).

We say $\AC$ is a \emph{(strong) localization of $\Cat{A}$ by 
$\Cat{C}$} if $T$ factors any $F: \Cat{A} \to \Cat{B}$ 
annihilating $\Cat{C}$ (i.e. $T$ solves Problem 1).
\end{definition}

The bulk of this section will be devoted to proving the following
theorem:

\begin{thm}[Serre]
If $\Cat{A}$ is well-powered (cf. \ref{def_well_powered}) then
for any Serre subcategory $\Cat{C}$, there exists a strong 
localization $\AC$ of $\Cat{A}$ by $\Cat{C}$.
\end{thm}

The theorem will be proved in the course of the section.

To define $\AC$, set $ob \AC = ob \Cat{A}$. 

\begin{rmk}
One way to define morphism, which the text does not adopt, is to
set $\hom_{\AC}(A, B) = \dlim \hom_{\Cat{A}}(A', B')$ where $\dlim$
is taken over
\[
\begin{diagram}
0 & \rTo & A'      & \rTo A \\
  &      & \dTo{f} \\
B & \rTo & B'      & \rTo 0
\end{diagram}
\]
where the rows are exact and $\Cat{C}$-isos, and $f$ and $f'$ are
identified in the limit if
\[
\begin{diagram}
0 & \rTo & A'      & \rTo & A''      & \rTo & A \\
  &      & \dTo{f} &      & \dTo{f'} \\
B & \rTo & B'      & \rTo & B''      & \rTo & 0.
\end{diagram}
\]
When assuming $\Cat{C}$ is small, this approach is correct (see 
Goursat's Lemma -- \ref{lemma_2_2} below), though the morphisms 
defined are difficult to work with. As Swan correctly points out,
the difficulty here is in defining composition and proving 
associativity.
\end{rmk}

Noting that any $f: A \to B$ can be factored as a $A 
\stackrel{(1, f)}{\to} A \times B \stackrel{\pi_B}{\to} B$, we adopt
a different strategy by considering the following:

\begin{definition}
Let $A, B \in \Cat{A}$. Then a \emph{pre-$\Cat{C}$ map} from $A$ to
$B$ is a subobject $\Gamma$ of $A \times B$ such that the composition
$\Gamma \to A \times B \to A$ is a $\Cat{C}$-iso.
\end{definition}

\begin{rmk}
The idea here is to define $\hom_{\AC}(A, B)$ to be the set of
pre-$\Cat{C}$ maps from $A$ to $B$. However, we must use caution:
Notice that $\Gamma'$ a subobject such that $\Gamma' \to \Gamma$ 
a $\Cat{C}$-iso gives rise to the same map in $\AC$.
\end{rmk}

Consider the symmetric relation on subobjects of $A \times B$ 
generated by ``$\Gamma'$ is a subobject of $\Gamma$ and
$\Gamma' \to \Gamma$ is a $\Cat{C}$-iso. That is, $\Gamma \sim
\Gamma'$ if there exists a sequence $\Gamma_1, \dots,
\Gamma_N$ together with ``zip-zap'' of subobjects
\[
\begin{diagram}
\Gamma = \Gamma_1 &&&& \dots &&&& \Gamma_N = \Gamma' \\
& \luTo & & \ruTo & & \luTo && \ruTo \\
&& \Gamma'_1 &&&& \Gamma'_{N - 1}
\end{diagram}
\]
where the maps $\Gamma'_i \to \Gamma_i$, $\Gamma'_i \to 
\Gamma_{i + 1}$ are $\Cat{C}$-isos. This is the weakest 
equivalence relation generated by the above relation.

We define $\hom_{\AC}(A, B)$ to be the equivalence class under 
$\sim$. Since $\Cat{A}$ is well-powered, $\hom_{\AC}(A, B)$ is a
set.


\begin{lem}[Goursat's Lemma]\label{lemma_2_2}
There exists a 1-to-1 correspondence between subojects of $A \times B$ and maps
of subojects of $A$ to quotients of $B$.
\end{lem}


