\section{Localization}

\noindent For the following, let $\Cat{A}$ be an abelian 
category, and let $\Cat{C}$ be a Serre subcategory. The following 
sections will deal with the following two questions:

\noindent {\bf Problem 1.} Find an exact functor $T$ and an 
abelian category $\Cat{A}/\Cat{C}$ with $T: \Cat{A} \to 
\Cat{A}/\Cat{C}$ such that if $F: \Cat{A} \to \Cat{B}$ is an 
exact functor which annihilates $\Cat{C}$ then there exists a 
unique functor $G$ with $F = GT$; $G$ will be exact. This is 
possible if $\Cat{A}$ is well-powered (proven in the next 
section) or if $\Cat{C}$ is localizing.

\vskip 10pt

\noindent {\bf Problem 2.} Find an exact functor $T$ from 
$\Cat{A}$ to some category $\Cat{D}$ which is universal for exact 
functors which annihilate $\Cat{C}$ up to natural isomophism. 
That is, given the diagram
\[
\begin{diagram}
   &          & D           \\
   & \ruTo{T} & \dDashto{G} \\
 A \\
   & \rdTo{F} \\
   &          & B
\end{diagram}
\]
where $F$ is exact and annihilates $C$. Then $G: \Cat{D} \to
\Cat{B}$ is exact and $GT$ is naturally isomorphic to $F$
and any other $G'$ is naturally isomorphic to $G$.

We solve the problems first for $\Cat{C}$ localizing.

\vskip 10pt
\begin{prop} \label{prop_2_0}
If $\Cat{C}$ is localizing, then $\Cat{L}$ and the reflection
$R: \Cat{A} \to \Cat{L}$ solves Problem 2.
\end{prop}
\begin{proof}
Let $F: \Cat{A} \to \Cat{B}$ be exact and annihilate $\Cat{C}$.
Set $G = F|_{\Cat{L}}$. Certainly the following commutes:
\[
\begin{diagram}
   &          & \Cat{L}     \\
   & \ruTo{T} & \dDashto{G} \\
 A \\
   & \rdTo{F} \\
   &          & B
\end{diagram}
\]
We must show that $G$ is exact. For s.e.s. in $\Cat{L}$,
\[
0 \to L' \to L \to L'' \to 0
\]
we have
\[
0 \to L' \to L \to L'' \to C \to 0
\]
in $\Cat{A}$, and applying $R$, note that $C \in \Cat{C}$;
but $\Cat{C}$ is annihilated by $F$. Therefore, $FC = 0$,
and $G = F|_{\Cat{L}}$ is exact.

Next, we show $GR$ is naturally iso to $F$. For each $A \in
\Cat{A}$, we have a $\Cat{C}$-iso $u_A : A \to RA$, which
induces an iso $FA \to FRA$ which is natural in $A$. As
$FRA = GRA$, $FA \to GRA$ is a natural iso.

Finally, we show that $F|_{\Cat{C}}$ is universal. Fix a
natural iso $\nu : GR \to F$. Then $\nu|_{\Cat{L}}$ is a
natural isomorphism from $GR|_{\Cat{L}} \to F|_{\Cat{L}}$.
But $GR|_{\Cat{L}} = G$.
\end{proof}

\begin{rmk}
I have yet to give a satisfactory explanation to the following 
passage, taken directly from Swan's notes:

\begin{quotation}
It could happen that $\Cat{A}$ was very big and $\Cat{L}$ small 
(by creating a new equivalent category with lots of objects in 
each is isomorphism type of $\Cat{A}$ not in $\Cat{L}$). Then the 
above construction won't work for problem 1 because $R$ will 
identify too many objects to be able to factor $F$ as $GR$. This
is the only difficulty. We remove it by the following construction.
\end{quotation}

My thoughts: Consider $A$ and its $\Cat{C}$-envelope $RA$; by 
definition, $RRA = RA$, and therefore $A$ necessarily maps to the 
same object as $RA$ via $GR$. Conversely, $FA$ may not be 
identified as $FRA$. Thus, necessarily, $GR \neq F$.

To make the above discussion concrete, we produce the following
$F$ as an example of the above, and also a solution to Problem 1.
\end{rmk}

Let $\AC$ be the category whose objects are objects of $\Cat{A}$
and 
\[
\hom_{\AC}(A, B) = \hom_{\Cat{A}}(RA, RB).
\]
There is a functor $T: \Cat{A} \to \AC$ defined by $TA = A$ and 
$Tf = Rf$. Let $S : \AC \to \Cat{L}$ be the functor given by $A 
\mapsto RA$ and $Sf = f$. It is clear that $R$ is an equivalence 
with inverse $S$.

It follows that $\AC$ is an abelian category that annihilates
$\Cat{C}$. Indeed, for $A \in \Cat{A}$, $RA = 0$ if and only if
$A \in \Cat{C}$. We also have the following:

\begin{thm}
The category $\AC$ defined as above solves problem 1 for 
$\Cat{A}$ with a localizing Serre subcategory $\Cat{C}$.
\end{thm}

\begin{proof}
$T$ is exact since $R$ is exact.

To show that $T$ is universal, consider $F : \Cat{A} \to \Cat{B}$
an exact functor that annihilates $\Cat{C}$, we want $G$ such that
$GT = F$. If such a $G$ exists, then it will be exact since $GSR$
is naturally iso to $F$, and $GS$ is exact by Proposition 
\ref{prop_2_0}; but $S$ is an equivalence.

Define $G = F$ on objects.

Next, consider $f \in \hom_{\AC}(A, B)$. This corresponds to a 
diagram in $\Cat{A}$:
\begin{equation}\label{2_star}
\begin{diagram}
A         &         & B         \\
\dTo{u_A} &         & \dTo{u_B} \\
RA        & \rTo{f} & RB
\end{diagram}
\end{equation}
Applying $F$ to the above, we have
\[
\begin{diagram}
FA         &          & FB         \\
\dTo{Fu_A} &          & \dTo{Fu_B} \\
FRA        & \rTo{Ff} & FRB
\end{diagram}
\]
Note that $Fu_A, Fu_B$ are isos, since $u_A, u_B$ are 
$\Cat{C}$-isos. Set 
\[
G(f) = (Tu_B)^{-1}F(f)Tu_A.
\]
First, consider
$u_A : A \to RA$. We want to show that 
\[
GTu_A = Fu_A.
\]
To see this, notice that $RA = RRA$ so $Ru_A = \id_{RA} = 
R\id_{RA}$. We have the following diagram associated to
$Tu_A$:
\[
\begin{diagram}
A         &            & RA             \\
\dTo{u_A} &            & \dTo{\id_{RA}} \\
RA        & \rTo{Tu_A} & RRA
\end{diagram}
\]
From which we derive
\[
\begin{diagram}
FA         &           & FRA       \\
\dTo{Fu_A} &           & \dTo{\id} \\
FRA        & \rTo{\id} & FRA
\end{diagram}
\]
And it is clear that $GTu_A = Fu_A$.

And similarly, applying $T$ to (\ref{2_star}), we have
\[
\begin{diagram}
TA         & \rTo{f}  & TB         \\
\dTo{Tu_A} &          & \dTo{Tu_B} \\
TRA        & \rTo{Tf} & TRB
\end{diagram}
\]

Applying $G$ to the above, we have
\[
\begin{diagram}
FA         & \rTo{Gf}  & FB         \\
\dTo{Fu_A} &           & \dTo{Fu_B} \\
FRA        & \rTo{GTf} & FRB
\end{diagram}
\]
and since $Fu_A$ and $Fu_B$ are isomorphisms, we have 
\[
GTf = Fu_BGf(Fu_A)^{-1} = Ff.
\]
Similar argumens show that $GTf = Ff$
for any $f \in \Cat{A}$.
\end{proof}

\begin{definition}
Fix $\Cat{A}$ an abelian category, and let $\Cat{C}$ be a Serre
subcategory. Suppose $\AC$ is any category with an exact functor 
$T: \Cat{A} \to \AC$. 

We say that $\AC$ is a \emph{weak localization of $\Cat{A}$ by 
$\Cat{C}$} exact functors annihilating $\Cat{C}$ (i.e. $T$ solves 
Problem 2).

We say $\AC$ is a \emph{(strong) localization of $\Cat{A}$ by 
$\Cat{C}$} if $T$ factors any $F: \Cat{A} \to \Cat{B}$ 
annihilating $\Cat{C}$ (i.e. $T$ solves Problem 1).
\end{definition}

The bulk of this section will be devoted to proving the following
theorem:

\begin{thm}[Serre]
If $\Cat{A}$ is well-powered (cf. \ref{def_well_powered}) then
for any Serre subcategory $\Cat{C}$, there exists a strong 
localization $\AC$ of $\Cat{A}$ by $\Cat{C}$.
\end{thm}

The theorem will be proved in the course of the section.

To define $\AC$, set $ob \AC = ob \Cat{A}$. 

\begin{rmk}
One way to define morphism, which the text does not adopt, is to
set $\hom_{\AC}(A, B) = \dlim \hom_{\Cat{A}}(A', B')$ where $\dlim$
is taken over
\[
\begin{diagram}
0 & \rTo & A'      & \rTo A \\
  &      & \dTo{f} \\
B & \rTo & B'      & \rTo 0
\end{diagram}
\]
where the rows are exact and $\Cat{C}$-isos, and $f$ and $f'$ are
identified in the limit if
\[
\begin{diagram}
0 & \rTo & A'      & \rTo & A''      & \rTo & A \\
  &      & \dTo{f} &      & \dTo{f'} \\
B & \rTo & B'      & \rTo & B''      & \rTo & 0.
\end{diagram}
\]
When assuming $\Cat{C}$ is small, this approach is correct (see 
Goursat's Lemma -- \ref{lemma_2_2} below), though the morphisms 
defined are difficult to work with. As Swan correctly points out,
the difficulty here is in defining composition and proving 
associativity.
\end{rmk}

Noting that any $f: A \to B$ can be factored as a $A 
\stackrel{(1, f)}{\to} A \times B \stackrel{\pi_B}{\to} B$, we adopt
a different strategy by considering the following:

\begin{definition}
Let $A, B \in \Cat{A}$. Then a \emph{pre-$\Cat{C}$ map} from $A$ to
$B$ is a subobject $\Gamma$ of $A \times B$ such that the composition
$\Gamma \to A \times B \to A$ is a $\Cat{C}$-iso.
\end{definition}

\begin{rmk}
The idea here is to define $\hom_{\AC}(A, B)$ to be the set of
pre-$\Cat{C}$ maps from $A$ to $B$. However, we must use caution:
Notice that $\Gamma'$ a subobject such that $\Gamma' \to \Gamma$ 
a $\Cat{C}$-iso gives rise to the same map in $\AC$.
\end{rmk}

\begin{definition}
We say $\Gamma$ and $\Gamma'$ are \emph{$\Cat{C}$-equivalent} if
$\Gamma$ and $\Gamma'$ are equivalent under the equivalence 
relation on subobjects of $A \times B$ generated by ``$\Gamma'$ 
is a subobject of $\Gamma$ and $\Gamma' \to \Gamma$ is a 
$\Cat{C}$-iso. That is, $\Gamma \sim \Gamma'$ if there exists a 
sequence $\Gamma_1, \dots, \Gamma_N$ together with ``zig-zag'' of 
subobjects
\[
\begin{diagram}
\Gamma = \Gamma_1 &&&& \dots &&&& \Gamma_N = \Gamma' \\
& \luTo & & \ruTo & & \luTo && \ruTo \\
&& \Gamma'_1 &&&& \Gamma'_{N - 1}
\end{diagram}
\]
where the maps $\Gamma'_i \to \Gamma_i$, $\Gamma'_i \to 
\Gamma_{i + 1}$ are $\Cat{C}$-isos. This is the weakest 
equivalence relation generated by the above relation.
\end{definition}

We define $\hom_{\AC}(A, B)$ to be the equivalence class under 
$\sim$. Since $\Cat{A}$ is well-powered, $\hom_{\AC}(A, B)$ is a
set.

\begin{lem}[Goursat's Lemma]\label{lemma_2_2}
There exists a 1-to-1 correspondence between subojects of $A 
\times B$ and maps of subojects of $A$ to quotients of $B$.
\end{lem}
\begin{proof}
\noindent Subobject to maps: Fix $F$ subobject of $A \times B$, 
and let $S$ be the image of $F \to A$, and $K$ the kernel of $F 
\to S$. We have
\[
\begin{diagram}
0 &\rTo & K &\rTo & F           & \rTo & S & \rTo & 0 \\
  &     &   &     & \dTo{\pi_B} \\
  &     &   &     & B           \\
  &     &   &     & \dTo        \\
  &     &   &     & \cok f
\end{diagram}
\]
Since $K \to F \to B$ is the map $f$, there is a map from $S \to 
\cok f$ (here $S$ is a subobject of $A$, and $\cok f$ is a 
quotient of $B$).

\noindent Maps to subobject: fix $0 \to S \to A$ and $B \to Q \to 
0$, with $f: S \to Q$, then consider the pullback diagram
\[
\begin{diagram}
P    & \rTo         & S \times B \\
\dTo &              & \dTo       \\
S    & \rTo{(1, f)} & S \times Q
\end{diagram}
\]
Check that these are inverses to each other.
\end{proof}

\begin{definition}
A \emph{relation between $A$ and $B$} is a subobject of $A \times 
B$. Relations are usually denoted $\rel{R}, \rel{S}$, etc.
\end{definition}

It is possible to ``compose'' relations. Say $\rel{R}$ is a 
relation between $A$ and $B$, and $\rel{S}$ is a relation
between $B$ and $C$, then define $\rel{S} \comp \rel{R}$ to be
the composition $P \to A \times B \times C \to A \times C$
where $P$ is the pullback of
\[
\begin{diagram}
P                & \rTo & \rel{R} \times C   \\
\dTo             &      & \dTo               \\
A \times \rel{S} & \rTo & A \times B \times C
\end{diagram}
\]
Similarly, we can define $\rel{R}^{-1}$ to be the composition of 
$\rel{R} \to A \times B \stackrel{\simeq}{\to} B \times A$.

\begin{definition}\label{relation_inverse}
For $\rel{R}$ and $\rel{S}$ relations between $A$ and $B$, we call
$\rel{R} \comp \rel{S}$ \emph{the composition of $\rel{R}$ and 
$\rel{S}$}. We call $\rel{R}^{-1}$ \emph{the inverse of $\rel{R}$}.
\end{definition}

\begin{definition}
We say $\rel{R}'$ is a subrelation of $\rel{R}$ (written 
$\rel{R}' \leq \rel{R}'$)if $\rel{R}$ is a subobject of $\rel{R}'$
\end{definition}

The following are clear consequences of the above (albeit somewhat
tedious to verify with the universal properties, or, if forfeiting
categorical purity is acceptable, reasonably straightforward via 
Embedding Theorem):

\begin{prop}\label{prop_2_47}
For $\rel{R}, \rel{S}$ and $\rel{T}$, and with $\rel{R}' \leq 
\rel{R}$, we have the following (whenever composition is defined)
\begin{enumerate}
\item $(\rel{S} \circ \rel{R})^{-1} = \rel{R}^{-1} \circ 
\rel{S}^{-1}$.

\item $\rel{S} \comp \rel{R}' \leq \rel{S} \comp \rel{R}$

\item $\rel{R}' \comp \rel{T} \leq \rel{R} \comp \rel{T}$

\item $\rel{R}'^{-1} \leq \rel{R}^{-1}$

\item $\rel{T} \comp (\rel{S} \comp \rel{R}) = (\rel{T} \comp 
\rel{S}) \comp \rel{R}$.
\end{enumerate}
\end{prop}

For $X \in A$, and relation $\rel{R}$ we define $\rel{R}(X)$ to 
be the image of the pullback $P$ in $B$ of the following:
\[
\begin{diagram}
P & \rTo & X \times B \\
\dTo &   & \dTo \\
\rel{R} & \rTo & A \times B \\
    &  &  &  \rdTo \\
    &  &  &   & B
\end{diagram}
\]

It is clear that
\begin{prop}
For $\rel{R}$ from $A$ to $B$ and $\rel{S}$ from $B$ to $C$, and 
for $X$ a subobject of $A$,
\[
\rel{S}(\rel{R}(X)) = (\rel{S} \comp \rel{R})(X)
\]
\end{prop}

\begin{definition}
Suppose $\rel{R}$ is a relation from $A$ to $B$, and $X$ a subobject
of $A$. We say $\rel{R}(X)$ is the \emph{image of $X$ under $\rel{R}$}.
The \emph{domain of $\rel{R}$} is $\rel{R}^{-1}(B)$, and the 
\emph{codomain of $\rel{R}$} is $\rel{R}(A)$.
\end{definition}

\begin{definition}
We write the diagonal relation $\Delta_A \to A \times A$ as 
$\rel{Id}_A$ (or simply $\rel{Id}$ if $A$ is understood). 
For $\rel{R}$ a relation between $A$ and $B$, we write 
$\rel{Id}_\rel{R}$ as the relation $\rel{R}^{-1} \comp \rel{R}$ 
as relation between $A$ and $A$.
\end{definition}

\begin{prop}
For relation $\rel{R}$ from $A$ to $B$ with domain $D$, we have 
$\rel{Id}_\rel{R} = \Delta_D$, where $\Delta_D$ is (image of) the 
diagonal embedding of $D$ in $A$.
\end{prop}

The following identifies the pre-$\AC$ maps in the set of relations
from $A$ to $B$.

\begin{lem}\label{lemma_2_3}
Let $\rel{R}$ be a relation between $A$ and $B$. Then $\rel{R}$ is
a pre-$\AC$ map if and only if
\begin{enumerate}
\item $X$ is a subobject of $A$ and $X \in \Cat{C}$ implies 
$\rel{R}(X) \in \Cat{C}$

\item $Y$ is a subobject of $B$ and $B/Y \in \Cat{C}$ implies
$A/\rel{R}^{-1}(Y) \in \Cat{C}$
\end{enumerate}
\end{lem}
\begin{proof}
\noindent $\Rightarrow$: Let $\rel{R}$ be a pre-$\AC$ map from $A$
to $B$. First, let $X$ be a suboboject of $A$, and form the pullback
$P$ in the following
\[
\begin{diagram}
     &      & K       \\
     &      & \dTo    \\
P    & \rTo & \rel{R} & \rTo & B \\
\dTo &      & \dTo    \\
X    & \rTo & A
\end{diagram}
\]
and let $K$ be the kernel of $\rel{R} \to A$, which is a 
$\Cat{C}$-iso, and so $K \in \Cat{C}$. Via $K \stackrel{0}{\to} X$,
we have a map $K \to P$. Since $X \in \Cat{C}$, $P \in \Cat{C}$,
and so is any image of $P \in B$ (since it is a quotient of $P$).

Second, suppose $Y$ is a subobject of $B$ and assume $B/Y \in 
\Cat{C}$. Form pullback and complete to s.e.s 
\[
\begin{diagram}
0 & \rTo & Q    & \rTo & \rel{R} & \rTo & B/Y  & \rTo 0 \\
  &      & \dTo &      & \dTo    &      & \dTo \\
0 & \rTo & Y    & \rTo & B       & \rTo & B/Y  & \rTo 0 \\
\end{diagram}
\]
Since $B/Y \in \Cat{C}$, $Q \to \rel{R}$ is a $\Cat{C}$-iso, and 
so is $Q \to \rel{R} \to A$. Thus $\cok (Q \to A) = 
A/\rel{R}^{-1} \in \Cat{C}$.

\noindent $\Leftarrow$: Assume now that (1) and (2) holds true.
Applying (1) to $0$, we have that $R(0) \in \Cat{C}$. Consider
the pullback diagram
\[
\begin{diagram}
0 & \rTo & P    & \rTo & \rel{R} \\
  &      & \dTo &      & \dTo    \\ 
0 & \rTo & 0    & \rTo & A       
\end{diagram}
\]
which shows that $P$ is the kernel of $\rel{R} \to A$, and $P \to 
\rel{R} \to B$ is a mono (since the kernel is $\rel{R}^{-1}(\im P 
\to B) = 0$), wherein $\rel{R}(0) = \im P \to B$ is isomorphic to 
$P$. Thus, $\rel{R} \to A$ is a $\Cat{C}$-mono .

Now consider the pullback diagram
\[
\begin{diagram}
P    & \rTo^=   & R    & \rTo & A \\
\dTo &          & \dTo \\
B    & \rEquals & B
\end{diagram}
\]
which shows that $\rel{R}^{-1}(B) = \im P \to A$, and $\cok (P 
\to A) = A/\rel{R}^{-1}(B) \in \Cat{C}$ by (2). Thus $\rel{R} \to 
A$ is a $\Cat{C}$-epi.
\end{proof}

We have the following consequences, clear from Lemma 
\ref{lemma_2_3} and Proposition \ref{prop_2_47}

\begin{cor}\label{cor_3_2}
If $\rel{R}$ and $\rel{S}$ are pre-$\Cat{C}$ maps, then so is
$\rel{S} \comp \rel{R}$.
\end{cor}

\begin{lem}\label{lemma_2_5}
If $\rel{R}$ is $\Cat{C}$-equivalent to $\rel{R}'$ and $\rel{S}$ is 
$\Cat{C}$-equivalent to $\rel{S}'$, then $\rel{S} \comp \rel{R}$ is
$\Cat{C}$-equivalent to $\rel{S}' \comp \rel{R}'$.
\end{lem}
\begin{proof}
Follows directly from Proposition \ref{prop_2_47} (3) and induction
on length of the ``zig-zag''.
\end{proof}

\begin{prop}
For $\Cat{A}$ an abelian category, and $\Cat{C}$ a Serre 
subcategory, the category $\AC$ whose objects are those of 
$\Cat{A}$ and morphisms
\[
\hom_{\AC}(A, B) = \{\; \Gamma\; |\; \Gamma\;
\textrm{pre-}\Cat{C}\textrm{ map from }A\textrm{ to }B \;\}
\] 
forms a category.
\end{prop}
\begin{proof}
By Corollary \ref{cor_3_2} and Lemma \ref{lemma_2_5}, composition 
is well-defined. Associativity follows from Proposition 
\ref{prop_2_47} (5). The relation $\rel{Id}_A$ (which is a 
$\Cat{C}$-iso) gives a two-sided identity on $\hom_{\AC}(A, A)$.
\end{proof}

Set $T(A) = A$ and $T(f) = [\Gamma_f]$ where $\Gamma_f$ is the
pre-$\AC$ map defined by the graph of $f$. We check that $T$ has 
the desired properties. We proceed by proving the following three 
propositions:

\begin{prop}
Let $F: \Cat{A} \to \Cat{B}$ be an exact functor of abelian 
categories such that $F$ annihilates $\Cat{C}$. Then there exists
a functor $G$ such that $F = GT$.
\end{prop}
\begin{proof}
Set $G(A) = F(A)$. Then $F = GT$ on objects.

Next, let $\rel{R} \to A \times B$ be a pre-$\AC$-map. Since $F$ 
is exact, $F(\rel{R})$ is a subobject of $F(A \times B) = FA 
\times FB$. As $\rel{R} \to A$ is a $\Cat{C}$-iso and $F$ 
annihilates $\Cat{C}$, $F\rel{R}$ is isomorphic to $FA$. Taking 
$g$ to be the composition $FA \to F\rel{R} \to FA \times FB \to 
FB$, define $G(\rel{R}) = g$.

We check that $G$ is well-defined: for $\rel{R}' \leq \rel{R}$ 
with $\rel{R}' \to \rel{R}$ a $\Cat{C}$-iso, we have a triangle 
of $\Cat{C}$-isos:
\[
\begin{diagram}
\rel{R}' \\
\dTo & \rdTo \\
\rel{R}    & \rTo & A
\end{diagram}
\]
Applying $F$, we have
\[
\begin{diagram}
F\rel{R}' \\
\dTo      & \rdTo(4,2) \\
F\rel{R}  & \rTo & FA \times FB & \rTo & FA
\end{diagram}
\]
But $F\rel{R} \simeq F\rel{R}'$ and correspond to the same 
subobject of $FA \times FB$. Therefore, $F\rel{R}$ and 
$F\rel{R}'$ give rise to the same map $g$.

Finally, we check that $G$ behaves well with respect to 
composition: $G(\rel{S} \comp \rel{S}) = G\rel{S}G\rel{R}$.
Fix pre-$\Cat{C}$ maps $\rel{R} \to A \times B$ and $\rel{S} \to 
B \times C$. Then $\rel{S} \comp \rel{R}$ is the image of $P$
in $A \times C$ of the diagram:
\[
\begin{diagram}
P                & \rTo & A \times \rel{S}    \\
\dTo             &      & \dTo                \\
\rel{R} \times C & \rTo & A \times B \times C \\
                 &      & \dTo                \\ 
                 &      & A \times C
\end{diagram}
\]
Applying $F$, we see that
\[
\begin{diagram}
FP                  & \rTo & FA \times F\rel{S}     \\
\dTo                &      & \dTo                   \\
F(\rel{R} \times C) & \rTo & F(A \times B \times C) \\
                    &      & \dTo                   \\ 
                    &      & F(A \times C).
\end{diagram}
\]
Let $\rel{T}$ be the image of $P$ in $A \times C$. By
Corollary \ref{cor_3_2}, $\rel{T}$ is also a pre-$\Cat{C}$ map
between $A$ to $C$, and $F\rel{T} \to FC$ is given by
$G\rel{S}G\rel{R}$.
\end{proof}

\begin{lem}\label{lemma_2_8}
Suppose $f: A \to B$ in $\Cat{A}$ is a $\Cat{C}$-iso,
then $T(f)$ is an iso in $\AC$.
\end{lem}
\begin{proof}
Since $f$ is $\Cat{C}$-iso, $\Gamma_f$ is a pre-$\AC$ map
since $\Gamma_f \to A$ is an iso, and $\Gamma_f \to B$ is a
$\Cat{C}$-iso (Lemma \ref{lemma_2_3}). Consequently,
$\Gamma_f^{-1}$ is also a pre-$\Cat{C}$ map, and defines
$h \in \hom_{\AC}(A,B)$. Since $hT(f)$ is represented by
$\Gamma_f^{-1}\Gamma_f$, conclude that $h$ is the desired
inverse; similarly for $T(f)h$.
\end{proof}

\begin{lem}
Let $h: TA \to TB$ in $\AC$. Then there exists a diagram
\[
\begin{diagram}
A &          &    &          & B \\
  & \luTo{f} &    & \ruTo{g} \\
  &          & A'
\end{diagram}
\]
in $\Cat{A}$ with $f$ a $\Cat{C}$-iso such that
\[
\begin{diagram}
TA &           & \rTo{h} &           & TB \\
   & \luTo{Tf} &         & \ruTo{Tg} \\
   &           & TA'
\end{diagram}
\]
commutes.
\end{lem}
\begin{proof}

\end{proof}

\begin{prop}
Let $G$ and $G'$ be functors from $\AC$ to $\Cat{B}$ where 
$\Cat{B}$ is any category. If $GT = G'T$ then $G = G'$.
\end{prop}

\begin{prop}
Let $G$ and $G'$ be functors from $\AC$ to $\Cat{B}$ for any 
category $\Cat{B}$ with $GT$ naturally isomorphic to $G'T$,
then $G$ is naturally isomoprhic to $G'$.
\end{prop}
